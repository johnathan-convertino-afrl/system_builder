\begin{titlepage}
  \begin{center}

  {\Huge system builder library}

  \vspace{25mm}

  \includegraphics[width=0.90\textwidth,height=\textheight,keepaspectratio]{img/AFRL.png}

  \vspace{25mm}

  \today

  \vspace{15mm}

  {\Large Jay Convertino}

  \end{center}
\end{titlepage}

\tableofcontents

\newpage

\section{Usage}

\subsection{Introduction}

\par
System builder is a python library that brings other build systems under one execution flow. The idea isn't to replace cmake, fusesoc, or other build systems. Its to glue
them together in a single scripted process to generate a end result. This allows these parts to be built using systems that make sense for their end goals. Overall the goal
is to make complete system generation for a Flash chip, SDCARD or other target easier and effortless for the end user.
\par
The library has three seperate objects that help with this task. The first is a yaml parser
that takes two yaml files, and commands and projects file, that are parsed and combined to create a series of dictionaries and nested lists. This data is then given to the executor
which creates threads that lanuch the commands and monitors them for errors. The last, and unimplimented, is a dependency check object. This will check for all dependencies and
report if they are installed and how to install them if they are not.
\par
The executor will show each target project and have its own current status to show its own progress bar. This shows the time elapsed, percent complete, status, and name of current target being build. This can be
disabled if needed. One note the progress bar up to the Target name is fixed to 80 columns. Anthing smaller then 80 will fail. Anything over 80 will show more of the target name. Meaning,
if you have only 80 columns, the target name will be cut off on the screen.

\subsection{Dependencies}

\par
The following are the dependencies of the cores.

\begin{itemize}
  \item python 3.X
  \item python progressbar2
\end{itemize}

\subsection{Example Creator}
\par
The creator is responsable for instantiating the library, preparing the project, and launching the system builder library to build the project.
\begin{lstlisting}[language=Python]
from system.build import *

cmd_compiler = commandCompiler("projects.yml", "commands.yml")

cmd_compiler.create()

projects = cmd_compiler.getResult()

cmd_exec = commandExecutor(projects, args.dryrun)

cmd_exec.runProject()

\end{lstlisting}

\subsection{Example Commands Yaml file}
This yaml file is injested so commands are executed with certain steps for the final end product.
\begin{lstlisting}[language=XML]
fusesoc:
  cmd_0: ["__CHECK_SKIP__{_pwd}/output/hdl/{_project_name}/AFRL_project_veronica_axi_baremetal_1.0.0.bit"]
  cmd_1: ["fusesoc", "--cores-root", "{path}", "run", "--build", "--work-root", "output/hdl/{_project_name}", "--target", "{target}", "{project}"]
script:
  cmd_1: ["{exec}", "{file}", "{_project_name}", "{args}"]
gcc_riscv32:
  cmd_0: ["__CHECK_SKIP__{_pwd}/output/bin/riscv/bin/riscv32-unknown-elf-gcc"]
  cmd_1: ["make", "-C", "{_pwd}/{path}", "clean"]
  cmd_2: ["__CWD__{_pwd}/{path}", "./configure", "--prefix={_pwd}/output/bin/riscv", "--disable-linux", "--with-arch=rv32imac", "--with-abi-ilp32"]
  cmd_3: ["make", "-C", "{_pwd}/{path}", "-j", "8"]
openocd_riscv:
  cmd_0: ["__CHECK_SKIP__{_pwd}/output/bin/openocd/bin/openocd"]
  #cmd_1: ["__CWD__{_pwd}/{path}", "git", "apply", "--ignore-whitespace", "../patch/pmpregs.patch"]
  cmd_2: ["__CWD__{_pwd}/{path}", "./bootstrap"]
  cmd_3: ["__CWD__{_pwd}/{path}", "./configure", "--prefix={_pwd}/output/bin/openocd", "--enable-ftdi", "--enable-dummy", "--enable-jtag_vpi"]
  cmd_4: ["make", "-C", "{_pwd}/{path}", "-j", "8"]
  cmd_5: ["make", "install", "-C", "{_pwd}/{path}"]
fpga_baremetal_examples:
  cmd_0: ["__CHECK_SKIP__{_pwd}/output/apps"]
  cmd_1: ["mkdir", "-p", "{_pwd}/{path}/cmake"]
  cmd_2: ["__CWD__{_pwd}/{path}/cmake", "__ENV_PATH__{_pwd}/output/bin/riscv/bin/", "cmake", "../", "-DCMAKE_RUNTIME_OUTPUT_DIRECTORY={_pwd}/output/apps", "-DCMAKE_PREFIX_PATH={_pwd}/output/bin/riscv/bin/", "-DBUILD_EXAMPLES_ALL=ON", "-DCMAKE_TOOLCHAIN_FILE={_pwd}/{path}/arch/riscv/riscv.cmake"]
  cmd_3: ["__ENV_PATH__{_pwd}/output/bin/riscv/bin/", "make", "-C", "{_pwd}/{path}/cmake"]
buildroot:
  cmd_1: ["rm", "-rf", "{_pwd}/output/linux/{_project_name}"]
  cmd_2: ["make", "-C", "{path}", "distclean"]
  cmd_3: ["make", "O={_pwd}/output/linux/{_project_name}", "-C", "{path}", "{config}"]
  cmd_4: ["make", "O={_pwd}/output/linux/{_project_name}", "-C", "{path}"]
genimage:
  cmd_1: ["mkdir", "-p", "{_pwd}/output/genimage/tmp/{_project_name}"]
  cmd_2: ["genimage", "--config", "{path}/{_project_name}.cfg"]
\end{lstlisting}

\subsection{Example Projects Yaml file}
This takes the commands and fills out the information required and order of operations for creating the finaly end result project.
\begin{lstlisting}[language=XML]
zed_fmcomms2-3_linux_busybox_sdcard:
  concurrent:
    fusesoc: &fusesoc_fmcomms2-3
      path: hdl
      project: AFRL:project:fmcomms2-3:1.0.0
      target: zed_bootgen
    buildroot: &buildroot_fmcomms2-3
      path: sw/linux/buildroot-afrl
      config: zynq_zed_ad_fmcomms2-3_defconfig
  sequential:
    script: &output_files_fmcomms2-3
      exec: python
      file: py/output_gen.py
      args: "--rootfs output/linux --bootfs output/hdl --dest output/sdcard"
    genimage:
      path: img_cfg
zc702_fmcomms2-3_linux_busybox_sdcard:
  concurrent:
    fusesoc:
      <<: *fusesoc_fmcomms2-3
      target: zc702_bootgen
    buildroot:
      <<: *buildroot_fmcomms2-3
      config: zynq_zc702_ad_fmcomms2-3_defconfig
  sequential:
    script:
      <<: *output_files_fmcomms2-3
    genimage:
      path: img_cfg
\end{lstlisting}

\section{Code Documentation} \label{Code Documentation}

\par
Natural docs is used to generate documentation for this project. The next lists the following sections.

\begin{itemize}
\item \textbf{init} python init code\\
\item \textbf{builder} system builder library.\\
\item \textbf{creator} system builder example code.\\
\end{itemize}

